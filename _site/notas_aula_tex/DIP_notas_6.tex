\documentclass{article}

\usepackage{graphicx} % Required for inserting images
\usepackage{hyperref}
\usepackage[
    backend = biber,
    style = abnt,
    ]{biblatex}
\addbibresource{bibliografia.bib}
\usepackage[shortlabels]{enumitem}
\usepackage[brazilian]{babel}

\title{%
  Direito Internacional Público \\
  \Large Notas de Aula 6\\
  \Large O Estado: dimensão pessoal do Estado\\
  }
\author{Luccas Gissoni}
\date{15 de outubro de 2024}

\begin{document}

\maketitle
\tableofcontents

\section{Introdução: população e comunidade nacional}

\subsection{População}

População do Estado é o conjunto das pessoas instaladas em caráter permanente sobre seu território:

\begin{itemize}
    \item Nacionais;
    \item Estrangeiros residentes
\end{itemize}

\subsection{Comunidade nacional}

Comunidade nacional é conjunto de nacionais do Estado, residam estes no interior de seu território ou no estrangeiro.

\begin{quote}
    Importante lembrar que a dimensão pessoal do Estado soberano (seu elemento constitutivo, ao lado do território e do governo) não é a respectiva população, mas a comunidade nacional, ou seja, o conjunto de seus nacionais, incluindo aqueles, minoritários, que se tenham estabelecido no exterior. Sobre os estrangeiros residentes o Estado exerce inúmeras competências inerentes à sua jurisdição territorial. Sobre seus nacionais distantes o Estado exerce jurisdição pessoal, fundada no vínculo de nacionalidade, e independente do território onde se encontrem \cite[p.~79]{rezek_direito_2024}.
\end{quote}

\section{Nacionalidade}

\begin{quote}
    Nacionalidade é um vínculo político entre o Estado soberano e o indivíduo, que faz deste um membro da comunidade constitutiva da dimensão pessoal do Estado \cite[p.~79]{rezek_direito_2024}.
\end{quote}

A nacionalidade é importante conceito de direito internacional, mas recebe uma disciplina jurídica de direito interno: cada Estado legisla sobre sua própria nacionalidade, ressalvadas as regras gerais e específicas com as quais tenha se comprometido.

O titular da nacionalidade é a pessoa humana, que através dela tem vínculo político com o Estado nacional. A nacionalidade das pessoas jurídicas constitui uma sujeição de ordem administrativa, mutável com a compra e venda ou sede da empresa; a nacionalidade das coisas é meramente metafórica.

\subsection{A nacionalidade em direito internacional}

\subsubsection{Princípios gerais}

\begin{itemize}
    \item A dimensão humana ou pessoal é elemento inerente ao Estado: o Estado necessita manifestar-se em determinadas pessoas;
    \item Todo indivíduo tem direito a uma nacionalidade; o Estado não pode arbitrariamente privar o indivíduo de sua nacionalidade, nem do direito de mudar de nacionalidade (art. 15; Declaralção Universal dos Direitos do Homem/1948);
    \item Princípio da efetividade: o vínculo patrial não deve fundar-se na pura formalidade ou no artifício, mas na existência de laços sociais consistentes entre o indivíduo e o Estado
\end{itemize}

\begin{quote}
    De modo geral a \textit{nacionalidade originária} (aquela que a pessoa se vê atribuir quando nasce) resulta da consideração, em grau variado, do lugar do nascimento (\textit{jus soli}) e da nacionalidade dos pais (\textit{jus sanguinis}). A manifestação de vontade — que opera às vezes como elemento acessório para a determinação da nacionalidade originária — é pressuposto indispensável da aquisição ulterior de outro vínculo patrial, mas deve apoiar-se sobre fatos sociais indicativos da relação indivíduo-Estado. Com efeito, a \textit{nacionalidade derivada}, que se obtém mediante naturalização e na maioria dos casos implica a ruptura do vínculo anterior, há de ter requisitos como alguns anos de residência no país, o domínio do idioma, e outros mais, ora alternativos ora cumulativos. Quando um Estado concede a alguém sua nacionalidade por naturalização carente de apoio em fatos sociais, não se discute seu direito de prestigiar esse gracioso vínculo dentro de seu próprio território. Lá fora, contudo, outros governos, e destacadamente os foros internacionais, tenderão a negar reconhecimento a essa nacionalidade, considerada inefetiva \cite[p.~80]{rezek_direito_2024}.
\end{quote}

\subsubsection{Normas costumeiras}

\begin{itemize}
    \item Exclui-se da atribuição de nacionalidade \textit{jure soli} os filhos de agentes de Estados estrangeiros;
    \item É proibido o \textit{banimento}: ``Nenhum Estado pode expulsar nacional seu, com destino a território estrangeiro ou a espaço de uso comum. Há, pelo contrário, uma obrigação, para o Estado, de acolher seus nacionais em qualquer circunstância, incluída a hipótese de que tenham sido expulsos de onde se encontravam" \cite[p.~80]{rezek_direito_2024}.
\end{itemize}

\subsubsection{Tratados multilaterais}

\begin{quote}
    O direito internacional escrito tem, de modo esparso e avulso, procurado reduzir os problemas da apatria e da polipatria, ora trazendo à ordem geral certos Estados excessivamente absorventes, ou, pelo contrário, refratários demais à outorga da nacionalidade, ora tendendo a proscrever, nesse âmbito, a distinção entre os sexos e a repercussão automática do casamento, ou de sua dissolução, sobre o vínculo patrial \cite[p.~80]{rezek_direito_2024}.
\end{quote}

\begin{itemize}
    \item \textbf{Convenção da Haia} de 12 de abril de 1930: liberdade do Estado para determinar em direito interno quais são seus nacionais, ponderando, embora, que tal determinação só é oponível aos demais Estados quando revestida de um mínimo de efetividade, à base de fatores ditados pelo costume pertinente (lugar do nascimento, filiação, tempo razoá­vel de residência ou outro indicativo de vínculo como pressuposto da naturalização);
    \item \textbf{Convenção da Haia} de 12 de abril de 1930: condenação da repercussão de pleno direito sobre a mulher, na constância do casamento, da eventual mudança de nacionalidade do marido, e a determinar aos Estados cuja lei subtrai a nacionalidade à mulher em razão do casamento com estrangeiro, que se certifiquem da aquisição, por aquela, da nacionalidade do marido, prevenindo desse modo a perda não compensada, vale dizer a apatria;
    \item \textbf{7ª Conferência Interamericana}: condenação de qualquer legislação ou prática discriminatória sobre os direitos da mulher;
    \item \textbf{Convenção de Nova York} de 20 de fevereiro de 1957: imuniza a nacionalidade da mulher contra todo efeito automático do casamento, do divórcio, ou das alterações da nacionalidade do marido na constância do vínculo;
    \item \textbf{Assembleia Geral das Nações Unidas} de 1948: trazia a nacionalidade à área dos direitos funda­mentais da pessoa humana, tendo como premissa maior a consideração do desamparo e dos transtornos resultantes da apatria;
    \item \textbf{Declaração Universal dos Direitos do Homem}, art. 15: ``todo homem tem direito a uma nacionalidade”; ``ninguém será arbitrariamente privado de sua nacionalidade, nem do direito de mudar de nacionalidade”;
    \item \textbf{Convenção americana sobre direitos humanos de São José da Costa Rica} de em 1969, art. 20: repete o art. 15 da Declaração Universal dos Direitos do Homem e inova: ``Toda pessoa tem direito à nacionalidade do Estado em cujo território houver nascido, se não tiver direito a outra”.
\end{itemize}

\subsection{A nacionalidade brasileira}

\subsubsection{Brasileiros natos}

Brasileiro nato é aquele que, \textit{nascido no Brasil ou no exterior}, teve a nacionalidade brasileira atribuída \textbf{ao nascer} ou ``a perspectiva de consolidá-la mediante opção, de efeitos retroativos" \cite[p.~81]{rezek_direito_2024}.

\begin{itemize}
    \item CF/88, art. 12, I, a): \textbf{os nascidos na República Federativa do Brasil}, ainda que de pais estrangeiros, desde que estes não estejam a serviço de seu país;
    \begin{quote}
        Um problema vestibular, mais complicado do que se poderia à primeira vista supor, é o da noção do que seja território brasileiro. Seria prático compreendê-lo no mais estrito dos sentidos, ou seja, como a massa territorial contínua dividida em unidades federadas. Desse modo, nascer no Brasil significaria, necessariamente, ter a naturalidade fixada em uma das quase seis mil circunscrições municipais em que o solo pátrio se subdivide. De outro modo, ter-se-ia nascido fora do território brasileiro, o que excluiria o jus soli mas não a perspectiva de atribuição da nacionalidade originária por outra das fórmulas constitucionais. O constituinte nada disse acerca dos espaços hídricos, aéreos, ou mesmo terrestres, imunes a qualquer incidência de soberania (o alto-mar, o espaço aéreo, o continente antártico). Mas quem aí vem ao mundo também não pode ser considerado por nós como nascido no estrangeiro, visto que tais espaços são neutros, e de uso comum disciplinado pelo direito internacional. Transferido o problema à doutrina, Pontes de Miranda propôs solução convincente: entendem-se nascidos no Brasil os nascidos a bordo de navios ou aeronaves de bandeira brasileira quando trafeguem por espaços neutros. O mesmo não ocorre, obviamente, em espaços afetos à soberania de outro Estado, mesmo se público o engenho onde acontece o nascimento. Há também a hipótese de alguém nascer a bordo, no mar territorial brasileiro ou em nosso espaço aéreo, qualquer que seja a bandeira do navio ou aeronave. Esses espaços nada têm de estrangeiros ou de neutros, de modo que justificam a atribuição da nacionalidade jure soli. Mas é também de Pontes de Miranda a lembrança de que, em tais casos, é pouco provável que se reclame a nacionalidade brasileira se nenhum outro vínculo existe entre a família e o Brasil. Se assim não for, não há como recusar a nacionalidade brasileira originária, solicitada pelos genitores ou pelo próprio interessado no futuro. Vale recordar que nosso sistema não impõe preclusões, não impõe perda de direitos, quanto à nacionalidade, pelo decurso do tempo. Países existem em que a partir de certa idade (25 anos na França) já não é possível discutir esse tema nem modificar a nacionalidade originária consolidada. Entre nós, é juridicamente possível que alguém por aqui apareça já idoso, provando, entretanto, que nasceu em Viena, ou em Xangai, da união informal entre o cônsul do Brasil e uma nacional da terra; ou que nasceu a bordo de um engenho norte-americano em trânsito no espaço aéreo ou no mar territorial brasileiro; ou que nasceu em plena selva, junto da fronteira, quando sua mãe, colombiana, alcançava o território brasileiro numa fuga à polícia, ou à guerrilha, ou ao narcotráfico. Em qualquer dessas hipóteses, provada a materialidade do fato, o reconhecimento da nacionalidade originária se impõe \cite[pp.~81-82]{rezek_direito_2024}.
    \end{quote}
    \item A regra constitucional do \textit{jus soli} comporta exceção expressa em seu desfecho: não são brasileiros, embora nascidos no Brasil, os filhos de pais estrangeiros que aqui se encontrem a serviço de seu país. O serviço, desde que público e afeto a potência estrangeira, não precisa implicar permanência em nosso território, nem cobertura das imunidades diplomáticas;
    \begin{quote}
        Há, na exceção ao jus soli, outro aspecto relevante, em torno do qual os autores não discrepam: os pais, estrangeiros, devem estar a serviço do país cuja nacionalidade possuem para que não ocorra a atribuição da nacionalidade brasileira. Seria brasileiro, dessa forma, o filho de um egípcio que cuidasse no Brasil da representação de Catar ou de Omã. A quem estranhe essa particularidade, convém lembrar que o constituinte não tencionou abrir exceção ao jus soli senão quando em presença de uma contundente presunção de que o elemento aqui nascido terá outra na­cionalidade, merecedora, por razões naturais, de sua preferência, e de que assim a atribuição da nacionalidade local iria originar quase que seguramente uma incômoda bipatria, a seu tempo resolvida em favor da nacionalidade estrangeira. Mas, se o Estado patrial dos genitores não é aquele mesmo a cujo serviço se encontram, a presunção perde sua energia, de modo que a recusa da nacionalidade local jure soli poderia não raro dar origem a uma situação que a todo custo tem de ser evitada, qual seja a apatria de um natural do Brasil  \cite[p.~82]{rezek_direito_2024}.
    \end{quote}
    \item CF/88, art. 12, I, b): \textbf{os nascidos no estrangeiro, de pai brasileiro ou mãe brasileira, desde que qualquer deles esteja a serviço da República Federativa do Brasil};
    \item CF/88, art. 12, I, c): \textbf{os nascidos no estrangeiro de pai brasileiro ou de mãe brasileira, desde que sejam registrados em repartição brasileira competente ou venham a residir na República Federativa do Brasil e optem, em qualquer tempo, depois de atingida a maioridade, pela nacionalidade brasileira}
\end{itemize}

\subsubsection{Brasileiros naturalizados}

\begin{itemize}
    \item CF/88, art. 12, II, a): os que, na forma da lei, adquiram a nacionalidade brasileira, exigidas aos originários de países de língua portuguesa apenas residência por um ano ininterrupto e idoneidade moral;
    \item CF/88, art. 12, II, b): os estrangeiros de qualquer nacionalidade, residentes na República Federativa do Brasil há mais de quinze anos ininterruptos e sem condenação penal, desde que requeiram a nacionalidade brasileira
    \begin{quote}
        A Constituição do Brasil cuida, ela própria, de favorecer a naturalização dos imigrantes que se fixaram no país há mais de quinze anos, sem quebra de continuidade e sem condenação penal; bem assim a dos originários de países de língua portuguesa, aos quais se exige como prazo de residência no Brasil apenas um ano ininterrupto e idoneidade moral. Dos demais estrangeiros a lei ordinária exige, em regra, quatro anos de residência no Brasil, idoneidade, boa saúde e domínio do idioma. O requisito cronológico é atenuado em certas hipóteses, como a de casamento com pessoa local ou prestação de bons serviços ao país. \cite[pp.~82-83]{rezek_direito_2024}.
    \end{quote}
\end{itemize}

\subsubsection{Perda da nacionalidade brasileira}

\begin{itemize}
    \item ``A extinção do vínculo patrial pode atingir tanto o brasileiro nato quanto o naturalizado em caso de aquisição de outra nacionalidade, por naturalização voluntária. Nesta hipótese, em face da prova da naturalização concedida lá fora, o presidente da República se limita a declarar a perda da nacionalidade brasileira. Seu ato não tem caráter constitutivo, vale dizer, não é dele que deriva a perda, mas da naturalização, que o antecede, e por força da qual se rompe o primitivo vínculo, restringindo-se o chefe do governo, a posteriori, a dar publicidade ao fato consumado. Para que acarrete a perda da nossa nacionalidade, a naturalização voluntária, no exterior, deve necessariamente envolver uma conduta ativa e específica" \cite[p.~83]{rezek_direito_2024}.
    \begin{quote}
        Se, ao contrair matrimônio com um francês, uma brasileira é informada de que se lhe concede a nacionalidade francesa em razão do matrimônio, a menos que, dentro de certo prazo, compareça ela ante o juízo competente para, de modo expresso, recusar o benefício, sua inércia não importa naturalização voluntária. Não terá havido, de sua parte, conduta específica visando à obtenção de outro vínculo pátrio, uma vez que o desejo de contrair matrimônio é, por natureza, estranho à questão da nacionalidade. Nem se poderá imputar procedimento ativo a quem não mais fez que calar. Outra seria a situação se, consumado o matrimônio, a autoridade estrangeira oferecesse, nos termos da lei, à nubente brasileira a nacionalidade do marido, mediante simples declaração de vontade, de pronto reduzida a termo. Aqui teríamos autêntica naturalização voluntária, resultante de procedimento específico — visto que o benefício não configurou efeito automático do matrimônio —, e de conduta ativa, ainda que consistente no pronunciar de uma palavra de aquiescência  \cite[p.~83]{rezek_direito_2024}.
    \end{quote}
    \item ``O art. 12, § 4º, 2, b, da Constituição de 1988 ressalva a naturalização voluntária do brasileiro residente no exterior quando ela constitui, segundo o direito local, um pressuposto da simples permanência ou do mero exercício de direitos civis. Não ocorre mais, nessa hipótese, a perda da nacionalidade brasileira" \cite[p.~83]{rezek_direito_2024};
    \item ``O brasileiro naturalizado, e ele apenas, encontra-se sujeito a uma segunda espécie de medida excludente, qual seja o cancelamento da naturalização, por exercer atividade contrária ao interesse nacional. É óbvio que a variante implica processo capaz de comportar amplos meios de defesa" \cite[p.~83]{rezek_direito_2024}.
    \item ``À margem de todo esse complexo, cabe ao presidente da República anular, por decreto, a aquisição fraudulenta da qualidade de brasileiro. Não se trata, aqui, de uma hipótese de perda da nacionalidade: esta se entenderá nula, e, pois, inexistente desde o início. Ninguém pode perder algo que jamais tenha possuído a não ser em equívoca aparência" \cite[p.~83]{rezek_direito_2024}.
\end{itemize}

\printbibliography

\end{document}