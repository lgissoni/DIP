\documentclass{article}
\usepackage{graphicx} % Required for inserting images
\usepackage{hyperref}
\usepackage[
    backend = biber,
    style = abnt,
    ]{biblatex}
\addbibresource{bibliografia.bib}

\title{%
  Direito Internacional Público \\
  \Large Notas de Aula 3\\
  \Large Fontes do DIP/Tratados p. 1\\
  }
\author{Luccas Gissoni}
\date{03 de setembro de 2024}

\begin{document}

\maketitle
\tableofcontents

\section{Introdução: fontes do direito internacional público}

\begin{itemize}
    \item 1920: \textbf{Corte da Haia} (tribunal para resolver litígios entre estados sem qualquer limitação geográfica ou temática)
    \item Necessidade de se determinar as \textit{fontes} nas quais se poderia buscar as \textit{normas internacionais}
    \item \textbf{Fontes:} tratados, costumes e princípios gerais do direito
    \item \textbf{Meios auxiliares:} jurisprudência e doutrina
    \item Possibilidade do emprego da \textit{equidade}
\end{itemize}

Nesta e nas próximas aulas trataremos dos \textbf{tratados}.

\section{Perspectiva histórica}

\begin{itemize}
    \item Prática da celebração de tratados: muito antiga, de natureza \textit{costumeira} e assentada sobre os princípios do \textit{pacta sunt servanda} e da boa-fé
    \item Primeiro registro: tratato de paz celebrado entre Hatusil III, Rei dos hititas, e Ramsés II, Faraó do Egito, entre 1280 e 1272 a.C.
    \item Século XIX: sem perder a consistência costumeira, os tratados ampliam seu acervo normativo com a emergência da \textit{multilateralidade}
    \item Também nesse século, ocorre, com o \textit{republicanismo} e o \textit{constitucionalismo}, a erosão do protagonismo centrado no chefe de Estado
\end{itemize}

\begin{quote}
    Resultou induvidoso que essa \textit{fase interna}, a da consulta ao parlamento como preliminar de ratificação, impôs ao direito das gentes uma importante remissão ao direito doméstico dos Estados \cite[p.~12]{rezek_direito_2024}.
\end{quote}

\begin{itemize}
    \item \textbf{Convenção de Havana sobre tratados (1928):} primeira formalização sobre o tema; texto sumário ainda vigente entre oito países latinoamericanos, dentre os quais o Brasil
    \item \textbf{Convenção de Viena sobre o direito dos tratados (1969):} entrou em vigor em 1980, sendo ratificada pelo Brasil em 2009; a Convenção formaliza normas preexistentes de caráter costumeiro
\end{itemize}

\section{O fenômeno convencional}

\subsection{Conceito}

\begin{quote}
    Tratado é todo acordo formal concluído entre pessoas jurídicas de direito internacional público, e destinado a produzir efeitos jurídicos. Na afirmação clássica de Georges Scelle, o tratado internacional é em si mesmo um simples \textit{instrumento}; identificamo-lo por seu processo de produção e pela \textit{forma} final, não pelo conteúdo. Este — como o da lei ordinária em uma ordem jurídica interna — é variável ao extremo \cite[p.~12]{rezek_direito_2024}.
\end{quote}

\subsection{Terminologia}

A terminologia empregada é vasta: \textit{acordo, ajuste, arranjo, ata, ato, carta, código, compromisso, constituição, contrato, convenção, convênio, declaração, estatuto, memorando, pacto, protocolo e regulamento} etc.

\begin{quote}
    Esses termos são de uso livre e aleatório, não obstante certas preferências denunciadas pela análise estatística: as mais das vezes, por exemplo, carta e constituição vêm a ser os nomes preferidos para tratados constitutivos de organizações internacionais, enquanto ajuste, arranjo e memorando têm largo trânsito na denominação de tratados bilaterais de importância reduzida. Apenas o termo concordata possui, em direito das gentes, significação singular: esse nome é estritamente reservado ao tratado bilateral em que uma das partes é a Santa Sé, e que tem por objeto a organização do culto, a disciplina eclesiástica, missões apostólicas, relações entre a Igreja católica local e o Estado copactuante \cite[p.~13]{rezek_direito_2024}.
\end{quote}

Essa liberalidade no usto da terminologia não ocorre com alguns nomes compostos: por exemplo, \textit{acordo de sede} refere-se necessariamente a um ajuste entre uma organização internacional e um Estado para instalação física da sede da organização no território do referido Estado; já \textit{compromisso arbital} designa um ajuste por meio do qual dois Estados decidem submeter litígio entre eles à arbitragem internacional.

\subsection{Formalidade}

Tratado é um acordo \textit{formal} e nisto reside sua principal diferença em relação ao costume. Ela também implica na \textit{escritura}.

\subsection{Atores}

As partes de um tratado são \textit{pessoas jurídicas de direito internacional público}: Estados soberanos e organizações internacionais. Empresas privadas não tem personalidade jurídica de DIP e portanto tampouco capacidade para celebrar tratados.

\subsection{Efeitos jurídicos}

Tratados geram vínculos obrigacionais de natureza jurídica. Para isto, é necessária a efetiva vontade de criá-los (\textit{animus contrahendi}).

\begin{quote}
    A produção de efeitos de direito é essen­cial ao tratado, que não pode ser visto senão na sua dupla qualidade de ato jurídico e de norma. O acordo formal entre Estados é o ato jurídico que produz a norma, e que por produzi-la desencadeia efeitos de direito, gera obrigações e prerrogativas, caracteriza enfim, na plenitude de seus dois elementos, o tratado internacional \cite[p.~13]{rezek_direito_2024}.
\end{quote}

Os tratados devem, deste modo, ser distinguidos do acordo em sentido amplo, ainda que formalizado. Fala-se também no \textit{gentlemen's agreement}, isto é, o ``pacto pessoal \textit{entre estadistas}, fundado sobre a honra, e condicionado, no tempo, à permanência de seus atores no poder" \cite[p.~13]{rezek_direito_2024}.

\section{Crítica}

\begin{quote}

O MILS, como é seu costume, oferece uma definição formal de tratados. Ele geralmente remete
o leitor ao Artigo 2(1)(a) da Convenção de Viena sobre o Direito dos
Tratados de 1969, que define um tratado como ‘um acordo internacional concluído entre Estados em forma escrita e regido pelo direito
internacional, seja incorporado em um único instrumento legal ou em dois ou mais
instrumentos relacionados, e qualquer que seja sua designação particular’. 38 Há
pouca indicação nesta definição abstrata das relações sociais encapsuladas em um tratado. Compare isso com as definições oferecidas pelos estudiosos
soviéticos Korovin e Pashukanis: ‘Todo acordo internacional é a
expressão de uma ordem social estabelecida, com um certo equilíbrio de interesses
coletivos’; 39 ‘Uma obrigação de tratado não é nada mais do que uma forma especial de
concretização de relações econômicas e políticas.’ 40 Essas definições, por meio do uso da realidade extratextual, oferecem maior percepção
do significado de um tratado do que a definição formal oferecida pelo MILS. Elas
nos remetem tanto ao fato de uma ordem social (capitalista) estabelecida quanto

sua concretização como regras econômicas e políticas que incorporam um certo
equilíbrio de interesses coletivos (de classe).
Um tratado é alcançado na matriz das regras já existentes do
‘jogo do tratado’ que esclarecem o que é permitido e o que é questionável
no curso da negociação e adoção de acordos internacionais. Essas
regras, codificadas no ‘tratado de tratados’, ou seja, a Convenção de Viena
sobre o Direito dos Tratados de 1969, favorecem participantes poderosos no processo de elaboração de tratados. Primeiro, a Convenção de Viena não exige nada
mais do que a adesão formal às regras da democracia deliberativa. Assim,
ela não impede a coerção silenciosa dos estados — um fato que tende a ser
ignorado pelo MILS. 41 Em outras palavras, como Brilmayer aponta, ‘argumentos
baseados no consentimento são enganosamente simples’ como explicação para a natureza vinculativa
dos acordos. 42 Seu ‘poder teórico reside na sugestão de que
talvez nada realmente precise ser justificado’. 43 Mas, como é evidente para qualquer
observador de negociações internacionais, ‘a barganha frequentemente ocorre
em um mundo de recursos desiguais e custos de oportunidade’. 44 Ela fornece o
solo no qual a coerção silenciosa floresce. Segundo, há a questão de
quem está consentindo: o estado ou as pessoas que o constituem? O MILS não
aborda essa questão, pois a norma do ‘direito à governança democrática’

não exige que a democracia participativa seja institucionalizada. Nas
circunstâncias, o tratado frequentemente oculta os interesses de certas classes
sociais, mesmo quando o MILS finge que incorpora compromissos acordados de
diferentes ‘interesses nacionais’. Embora seja reconhecidamente difícil capturar a
dimensão de classe dos tratados na linguagem do direito internacional quando se trata do processo de elaboração de leis, 45 não há razão para que o MILS não possa

preocupa-se com a exclusão de alguns assuntos de interesse para classes
subalternas do processo de elaboração de tratados, 46 aborda a questão da ausência
de democracia substantiva em estados consentidores e endossa a ideia de
responsabilidade transnacional do estado (um assunto ao qual retornaremos mais tarde),
a fim de promover a causa de grupos marginais e oprimidos. 47
Foi apropriadamente observado que ‘tratados e instrumentos semelhantes a tratados
no final do século XX tornaram-se muito importantes
para o funcionamento da sociedade internacional para permanecer ou se tornarem
propriedade de qualquer disciplina ou subdisciplina’. 48 Pois não há na era
da globalização nenhum aspecto das relações internacionais que não seja regulado pelo direito internacional dos tratados. Portanto, exige abordagens
mais sofisticadas para os tratados do que as oferecidas pelo MILS. De acordo com Johnston, uma ‘mudança substancial no tratamento de tratados por parte da disciplina
legal deve ser premissa... com base na lógica funcional – não formal – . . . ’. 49 A visão funcionalista que ele propõe ‘enfatiza a relevância de
considerações inerentes ao contexto em que a questão ou problema ocorre, para que as normas legais sejam mantidas em equilíbrio com as realidades institucionais e
políticas’. 50 Também compartilhamos o entendimento de que é:
útil distinguir três tipos de consentimento, variando com o instrumento e
as circunstâncias de sua negociação: consentimento para ser juridicamente responsável
em um tribunal, consentimento para ser considerado operacionalmente responsável na arena diplomática e consentimento para ser moralmente vinculado aos olhos da opinião pública progressista. 51
CMILS entende o último como um chamado para a avaliação e reformulação de
um tratado à luz de seu impacto em certos grupos na vida social, a saber
a classe trabalhadora, mulheres, camponeses e sem-terra, crianças, povos indígenas e assim por diante. O CMILS, no entanto, vai além da abordagem funcionalista
para promover uma estratégia abrangente que promova os interesses
das classes subalternas, sem minar totalmente uma abordagem
orientada para regras. Os seguintes elementos, entre outros, constituem tal
estratégia.
Primeiro, o CMILS pede uma codificação adicional das regras da democracia deliberativa. A Convenção de Viena sobre o Direito dos Tratados precisa
ser alterada para incluir disposições que proíbam o uso de todas as formas de

coerção (por exemplo, coerção econômica e diplomática) em negociações internacionais. 52 A CMILS também, para promover a democracia deliberativa, exigiria uma maior representação das classes subalternas nas equipes de negociação enviadas pelos estados.
Segundo, a CMILS gostaria de introduzir uma forma de sistema de avaliação de impacto social
baseado em pessoas. Em apoio a tal processo, a CMILS exigiria,
inter alia, que os tratados fossem negociados e ratificados com a consulta e consentimento dos representantes eleitos do povo. Embora essa
ação ainda possa não impedir o consentimento a tratados que não sejam do interesse
das classes subalternas, ela ajudaria a tornar o processo mais transparente
e passível de dissidência e mobilização política.
Terceiro, quando se trata de implementação de tratados, o CMILS prescreveria um conjunto de ferramentas legais que ofereceriam aos estados dependentes e dominados
flexibilidade para implementar suas obrigações de uma maneira que salvaguardasse os
interesses das classes subalternas. Assim, por exemplo, no contexto da
OMC, o CMILS defenderia uma maior deferência às interpretações nacionais e implementação de regras em vez de um modo uniforme insistido
pelo Órgão de Solução de Controvérsias (DSB) da OMC. Pois o princípio da deferência nacional se traduz em maior autonomia para os estados em um momento em que
está sendo subvertido por uma rede de acordos internacionais. 53 Ele
permitiria que as classes subalternas pressionassem o estado a adotar
interpretações que salvaguardassem seus interesses.
Quarto, o CMILS exigiria o esclarecimento e o desenvolvimento de
regras de interpretação do direito internacional consuetudinário para revelar,
e, posteriormente, limitar, a influência do poder no exercício interpretativo. Seguindo Wittgenstein, o CMILS acredita que, uma vez que o significado de
uma palavra reside em seu uso na linguagem, não existe tal coisa como determinância/indeterminação fora do mundo das práticas sociais.54 Consequentemente,
o que deve ser problematizado são as práticas sociais (de classe) e não o
conceito abstrato de significado. Tal problematização ajudaria a focar
nas raízes sociais (de classe) da interpretação. Posteriormente, há uma necessidade de
desenvolver e usar ‘normas intersticiais’ que reconheçam as raízes sociais (de classe)
de interpretações conflitantes (como manifestadas, inter alia, na resistência a
interpretações particulares) e chegar ao fechamento após tomar conhecimento de

as consequências sociais de ambos. 55 As normas intersticiais podem ser legais
(princípio da boa-fé) ou morais (equidade na resolução de reivindicações conflitantes). 56
Quinto, o CMILS reexaminaria e destacaria a doutrina rebus sic stantibus
ou mudança material nas circunstâncias e a tornaria parte integrante do
conceito de um tratado equilibrado e justo, embora em sua forma consensual. 57 Isso
está em contraste com o MILS, que afirma que "nos tempos modernos, é acordado
que a regra se aplica apenas nas circunstâncias mais excepcionais, caso contrário
poderia ser usada como uma desculpa para fugir de todos os tipos de obrigações inconvenientes". 58
Essa visão ignora o fato de que estados dependentes e dominados podem
apenas recorrer à doutrina rebus sic stantibus para abordar o problema de
tratados injustos. Claro que, na análise final, a invocação da cláusula
dependeria tanto das consequências sociais de obedecer a uma regra quanto da
resistência coletiva a ela pelos povos afetados. Neste contexto, o direito internacional
dos direitos humanos é de relevância óbvia, e o CMILS sugere que ele
deve ser cada vez mais utilizado para dar suporte à invocação da doutrina rebus
    sic stantibus \cite[p.~66-70, tradução por IA]{chimni_outline_2008}.

\end{quote}

\printbibliography

\end{document}
