\documentclass{article}

\usepackage{graphicx} % Required for inserting images
\usepackage{hyperref}
\usepackage[
    backend = biber,
    style = abnt,
    ]{biblatex}
\addbibresource{bibliografia.bib}
\usepackage[shortlabels]{enumitem}
\usepackage[brazilian]{babel}

\title{%
  Direito Internacional Público \\
  \Large PAP 2\\
  \Large Caso Prático - Soberania, Território e População no Direito Internacional\\
  }
\author{Luccas Gissoni}
\date{30 de outubro de 2024}

\begin{document}

\maketitle

\section{Situação Fática}

O Estado de Pacifica, uma pequena nação insular, proclamou sua independência há cinco anos, após um plebiscito realizado em sua população que anteriormente fazia parte de um Estado maior, o Reino de Atlântida. Durante o plebiscito, 85% da população de Pacifica votou a favor da independência.

Após a declaração de independência, o Reino de Atlântida recusou-se a reconhecer a soberania de Pacifica, alegando que o plebiscito foi realizado sem seu consentimento formal. No entanto, alguns Estados, como Paralia e Montesia, reconheceram a independência de Pacifica e estabeleceram relações diplomáticas com o novo Estado.

Pacifica, com uma população de aproximadamente 200 mil habitantes e um território de 10 mil km², possui recursos naturais importantes, como petróleo e pesca abundante, o que aumentou o interesse de outras nações em sua estabilidade política. Recentemente, o Reino de Atlântida tem feito pressões na ONU e em outros foros internacionais para que Pacifica seja considerada uma "entidade separatista" e não um Estado soberano, argumentando que o território pertence legitimamente a Atlântida, com base em tratados históricos.

Paralia, Montesia e outros Estados apoiadores de Pacifica têm argumentado que a vontade da população, expressa democraticamente através do plebiscito, e o efetivo controle do território por Pacifica são suficientes para legitimar sua soberania e que Atlântida está violando o princípio da autodeterminação dos povos.

Diante desse impasse, o Conselho de Segurança da ONU está discutindo se deve intervir para manter a paz na região, e uma das questões centrais é se Pacifica deve ou não ser reconhecida como um Estado soberano de pleno direito.

\section{Tarefa}

    \begin{enumerate}[1)]
        \item Identifique e analise as três dimensões essenciais para a constituição de um Estado soberano no direito internacional (população, território e soberania).
        \begin{enumerate}[(a)]
            \item Pacifica preenche os requisitos para ser considerada um Estado soberano? Justifique sua resposta.
            \item A ausência de reconhecimento de Pacifica por parte de alguns Estados impede sua existência como Estado no cenário internacional?
        \end{enumerate}
        \item Autodeterminação dos povos x Integridade territorial:
        \begin{enumerate}[(a)]
            \item Como o direito internacional equilibra o princípio da autodeterminação dos povos com o princípio da integridade territorial dos Estados?
            \item No caso de Pacifica, qual princípio deve prevalecer? Fundamente sua resposta com base em tratados e jurisprudências internacionais.
        \end{enumerate}
        \item O papel do reconhecimento internacional:
        \begin{enumerate}[(a)]
            \item Qual a relevância do reconhecimento de Pacifica por outros Estados?
            \item O reconhecimento unilateral por Estados como Paralia e Montesia pode ser suficiente para Pacifica ser considerada um Estado? Há precedentes no direito internacional que podem embasar essa discussão?
        \end{enumerate}
    \end{enumerate}
Cada item vale 0,5 ponto.

\section{Objetivos da Atividade}

\begin{itemize}
    \item Desenvolver a capacidade de análise crítica dos alunos sobre a formação de Estados e os critérios do Direito Internacional Público.
    \item Estimular a reflexão sobre conflitos entre autodeterminação e integridade territorial, um tema atual e relevante.
    \item Incentivar a pesquisa e o uso de jurisprudências e tratados internacionais como fundamento para a solução de casos práticos.
\end{itemize}

\section{Critérios de Avaliação}

\begin{itemize}
    \item Clareza e coerência nas respostas.
    \item Capacidade de relacionar os princípios teóricos com a prática.
    \item Uso adequado de doutrina e jurisprudência internacional.
    \item Argumentação fundamentada nos princípios do Direito Internacional.
\end{itemize}

\section{Prazo e entrega}

Favor entregar a atividade até \textbf{12/11/2024} no email \href{mailto:luccas.gissoni@gmail.com}{luccas.gissoni@gmail.com} ou pessoalmente em sala, preferencialmente digitado.

\end{document}