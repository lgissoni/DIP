\documentclass{article}

\usepackage{graphicx} % Required for inserting images
\usepackage{hyperref}
\usepackage[
    backend = biber,
    style = abnt,
    ]{biblatex}
\addbibresource{bibliografia.bib}
\usepackage[shortlabels]{enumitem}
\usepackage[brazilian]{babel}

\title{%
  Direito Internacional Público \\
  \Large Notas de Aula 7\\
  \Large Organizações internacionais\\
  }
\author{Luccas Gissoni}
\date{12 de novembro de 2024}

\begin{document}

\maketitle
\tableofcontents

\section{Introdução}

As organizações internacionais (OIs) são um conjunto de entes extremamente heterogêneo. Varia \textit{quantitativamente}, segundo seu alcance geográfico, quadro de pessoal e orçamento, e \textit{qualitativamente}, segundo suas finalidades e objetivos.

\begin{quote}
    Seus objetivos variam, com efeito, entre a suprema ambição de uma ONU — manter a paz entre os povos, preservar-lhes a segurança, e fomentar, por acréscimo, seu desenvolvimento harmônico — e o modestíssimo desígnio de uma UPU, consistente apenas em ordenar o trânsito postal extrafronteiras \cite[p. 104]{rezek_direito_2024}.
\end{quote}

\section{Teoria geral}

\subsection{Personalidade jurídica}

A personalidade jurídica da organização internacional envolve a capacidade jurídica, conforme tratado pioneiro que constituiu a Organização Internacional do Trabalho (OIT) em 1919:

\begin{quote}
    \begin{center}
          Artigo 39\\
    \end{center}
    A Organização Internacional do Trabalho deve ter personalidade jurídica, e,
    precipuamente, capacidade para:\\
    a) adquirir bens, móveis e imóveis, e dispor dos mesmos;\\
    b) contratar;\\
    c) intentar ações \cite{oit_constituicao_1919}.
\end{quote}

O elemento mais importante da personalidade jurídica é a competência da organização para celebrar tratados em seu próprio nome. Daí decorre a personalidade jurídica como \textit{consequência}:

\begin{quote}
    Paul Reuter, atento à fase incipiente em que se encontrava a formulação de uma teoria geral das organizações internacionais, insistia em que a personalidade jurídica de direito das gentes não é a fonte da competência da organização, mas seu resultado182. Se os pactuantes — ainda que despreocupados de lavrar um dispositivo do gênero do art. 39 da Constituição da OIT — definem os órgãos da entidade projetada, assinalando--lhes competências próprias a revelar autonomia em relação à individualidade dos Estados-membros, então, a partir da percepção dessa estrutura orgânica, e a partir, sobretudo, da análise dessas competências, será possível afirmar que o tratado efetivamente deu origem a uma nova personalidade jurídica de direito internacional público \cite[p. 107]{rezek_direito_2024}
\end{quote}

\subsection{Órgãos}

Aparentemente todas as OIs são dotadas de ao menos dois órgãos que parecem indispensáveis à sua estrutura: a assembleia geral e a secretaria. Um terceiro órgão cumemente encontrado é o conselho permanente.

\subsubsection{Assembleia geral}

Local os todos os Estados-membros tenham voz e voto em condições igualitárias. Representa o corpo dotado da competência legislativa da entidade. Ela não é permanente, reunindo-se ordinariamente - em geral uma vez por ano - e convocada extraordinariamente quando o exigirem as ircunstâncias.

\subsubsection{Secretaria}

``Órgão de administração, de funcionamento permanente, integrado por servidores neutros em relação à política dos Estados-membros — particularmente à de seus próprios Estados patriais" \cite[p. 107]{rezek_direito_2024}. Há uma tendência a se priorizar o mérito no recrutamento dos pessoal subalterno neutro, e também de se buscar uma partilha numérica de postos entre os Estados-membro.

\subsubsection{Conselho permanente}

Encontrável sobretudo nas OIs de vocação política, tem funcionamento ininterrupto e competência executiva \cite{rezek_direito_2024}. Pode ser composto por representates de todos os Estados-membros, como na Organização dos Estados Americanos (OEA), ou de alguns deles, eleitos pela assembleia geral por prazo certo ou dotados de mandato permanente. O Conselho de Segurança da Organização das Nações Unidas (ONU) conjuga exatamente as duas formas.

\subsubsection{Outros órgãos}

\begin{quote}
    Em função do seu alcance e dos seus propósitos, a organização internacional pode ter estrutura mais ampla, dispondo de outros conselhos — como, na ONU, o Econômico e Social —, bem assim de órgãos técnicos, de órgãos judiciários — como a Corte da Haia, no quadro da ONU, ou as cortes de Estrasburgo e de Luxemburgo, no quadro da União Europeia —, e até mesmo de órgãos temporários, fadados à extinção quando tenham atendido a certa finalidade conjuntural \cite[p. 108]{rezek_direito_2024}
\end{quote}

\subsection{Sede da organização}

\begin{quote}
    Carentes de base territorial, as organizações internacionais precisam de que um Estado faculte a instalação física de seus órgãos em algum ponto do seu território. Essa franquia pressupõe sempre a celebração de um tratado bilateral entre a organização e o Estado, a que se dá o nome de \textit{acordo de sede} \cite[p. 110]{rezek_direito_2024}.
\end{quote}

Não necessariamente a sede da OI se encontra em um Estado-membro, como no caso da Sociedade nas Nações (SDN), percursora da ONU fundada em 1919, que se instalou em Genebra. Assim, negociou acordo de sede com a Suíça, um Estado não-membro.

O acordo de sede impõe ao Estado celebrante obrigações relativas aos privilégios garantidos à OI copactuante e aos representantes de outros Estados-membros (delegados à assembleia geral, membros de conselho etc.)

\begin{quote}
    Um pedido israelense de fechamento do escritório da Organização para a Libertação da Palestina em Genebra foi rejeitado, em 1978, pelo Conselho Federal Suíço, que invocou, a propósito, suas obrigações resultantes do acordo de sede firmado com a ONU em 1946. O escritório da OLP fora aberto em 1975, a pedido das Nações Unidas, cuja Assembleia Geral havia decidido convidar a entidade a participar, com o estatuto de observador, de conferências e demais trabalhos promovidos pela organização \cite[p. 110]{rezek_direito_2024}.
\end{quote}

\subsection{ Representação, garantias, imunidade}

\begin{quote}
   A organização não goza de privilégios apenas no seu lugar de sede. Ela se faz representar tanto no território de Estados-membros quanto no de Estados estranhos ao seu quadro. Seus representantes externos são integrantes da secretaria — vale dizer, do quadro de funcionários neutros — e gozam de privilégios semelhantes àqueles do corpo diplomático de qualquer soberania representada no exterior187. Por igual, suas instalações e bens móveis têm a inviolabilidade usual em direito diplomático.\\
   Problema distinto deste dos privilégios estabelecidos pelo direito diplomático (basicamente a Convenção de Viena de 1961) é o da imunidade da própria organização internacional à jurisdição brasileira, em feito de natureza trabalhista ou outro. A jurisprudência assentada no Supremo Tribunal Federal desde 1989188 somente diz respeito aos Estados estrangeiros, cuja imunidade, no passado, entendia-se resultante de “velha e sólida regra costumeira”, na ocasião declarada insubsistente. No caso das organizações internacionais essa imunidade não resultou essencialmente do costume, mas de tratados que a determinam de modo expresso: o próprio tratado coletivo institucional, de que o Brasil seja parte, ou um tratado bilateral específico.\\
   A imunidade da organização, em tais circunstâncias, não pode ser ignorada, mesmo no processo de conhecimento, e ainda que a demanda resulte de uma relação regida pelo direito material brasileiro \cite[p. 111]{rezek_direito_2024}.
\end{quote}

\subsection{Finanças da organização}

A receita das OIs resulta sobretudo da \textit{cotização} dos Estados-membros. Outras receitas são bastante secundárias e às vezes não cobrem nem o respectivo custo. As cotizações não são paritárias, mas obedecem à \textit{capacidade contributiva} de cada Estado-membro, levando-se em conta seu poder econômico relativo.

\begin{quote}
    Assim, no biênio 2019-2021, os Estados Unidos entraram com 22\% da receita. A China contribuiu com cerca de 12\%, seguindo-se — com arredondamento dos números — o Japão (8½ \%), a Alemanha, a França e o Reino Unido (6\%), a Itália (5\%), o Brasil, o Canadá e a Espanha (3\%), a Rússia (2,5\%) e a Austrália (2\%). Com cotas superiores a 1\%, a Coreia do Sul, o México, os Países Baixos, a Turquia e a Suíça fecham a lista dos principais contribuintes. Todos os demais membros da ONU, cento e setenta e cinco, contribuem com somas correspondentes a menos que 1\% da receita total, e trinta e nove deles estancam no piso, que é de 0,001\% (vinte e cinco mil dólares nos últimos anos), embora a modéstia de seus recursos em confronto com os das demais nações pudesse justificar, no cálculo exato, uma contribuição ainda menor \cite[p. 115]{rezek_direito_2024}.
\end{quote}

As despesas da OI consistem na folha de pagamento do pessoal da secretara, custeio e manutenção de instalações imobiliárias e equipamentos, e muitas vezes também no custeio de programas exteriores de assistência e financiamento.

\begin{quote}
    No parecer consultivo de 20 de julho de 1962, solicitado pela Assembleia Geral da ONU, a Corte da Haia definiu como despesas da organização, a serem custeadas por sua verba orçamentária, as resultantes das operações da Força de Urgência das Nações Unidas em Suez e no Congo, na década de 1950. Essas operações haviam sido determinadas por voto majoritário da Assembleia e do Conselho de Segurança, sendo que determinados países membros da organização resistiam à ideia de copatrociná-las financeiramente \cite[p. 115]{rezek_direito_2024}.
\end{quote}

\subsection{Admissão de novos membros}

A admissão de novos membros depende de três aspectos: ``as condições prévias do ingresso, vale dizer, os limites de abertura da carta aos Estados não membros" \cite[p. 115]{rezek_direito_2024}, a adesão à carta e a aceitação pelos Estados-membros.

\subsubsection{Limites de abertura do tratado institucional}

Os limites podem ter caráter geográfico: a Carta da OEA está aberta à adesão dos “Estados americanos”; na Liga Árabe, todo ``Estado árabe" pode tornar-se membro da Organização.

\begin{quote}
    Na Carta das Nações Unidas a matéria é disciplinada pelo art. 4º: o interessado deve ser um Estado pacífico, que aceite as obrigações impostas pela carta, e que se entenda capaz de cumpri-las e disposto a fazê-lo. Subordina-se a análise destes últimos pressupostos ao juízo da própria organização \cite[p. 116]{rezek_direito_2024}.
\end{quote}

\subsubsection{Adesão ao tratado institucional}

Esta é a condição fundamental de ingresso e também a que casa menos controvérsia: basta o Estado exprimir sua adesão. Esta deve ser integral, isto é, feita sem reservas, as quais não foram facultadas aos pactuantes originários.

\subsubsection{Beneplácito à adesão}

É dado pelo órgão competente da entidade. Na SDN, a competência era da Assembleia, que devia se manifestar por dois terços. ``Na União Europeia (UE) é o Conselho que, sob parecer da Comissão, deve assentir por unanimidade (...). O Pacto da Liga Árabe prescreve a submissão do pedido ao Conselho sem nada adiantar sobre o quorum, permitindo supor que o veto seja praticável" \cite[p. 116]{rezek_direito_2024}.

\begin{quote}
    A última década do século XX foi marcada por expressivo número de admissões no quadro das Nações Unidas. Eram, de um lado, os últimos remanescentes coloniais que acederam à condição de Estados independentes e, de outro, as soberanias resultantes de desmembramentos ocorridos na Europa \cite[p. 116]{rezek_direito_2024}.
\end{quote}

\printbibliography

\end{document}